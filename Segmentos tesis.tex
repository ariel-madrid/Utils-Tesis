\subsection{Revisión preliminar de la literatura}


En la actualidad, para la estimación de los parámetros estructurales de los discos protoplanetarios, se utiliza una técnica llamada MCMC, que se encarga de alcanzar iterativamente una distribución deseada mediante la exploración de un espacio de parámetros. En el ámbito astronómico, se emplea una serie de algoritmos necesarios para lograr dicho objetivo. En cada iteración, se lleva a cabo lo siguiente: utilizando los algoritmos de transferencia radiativa de RADMC3D, se simula el paso de los fotones a través de una estructura de polvo, proceso que permite generar una imagen del disco con parámetros estructurales definidos de forma aleatoria al inicio del proceso. Posteriormente, se compara el modelo creado con los datos reales para ajustar un nuevo conjunto de parámetros. Para esta comparación, se calcula una distribución chi-cuadrado, que se obtiene a través de la interpolación bilineal entre los datos simulados y reales del disco.

\cite{diaz2023} desarrolló un modelo basado en aprendizaje profundo que utiliza redes neuronales convolucionales para estimar 5 parámetros estructurales de discos protoplanetarios. Sin embargo, presenta dos limitaciones principales: (1) es útil para estimar masas de discos que se encuentran entre 0.001 y 0.2 masas solares, y (2) ha sido entrenado únicamente con modelos de discos dentro de la banda 6 de ALMA. 



\section{Revisión de la literatura}

En cuanto al problema de la estimación de parámetros estructurales de discos protoplanetarios existen estudios previos que han realizado intentos de mejorar aquellos basados en las cadenas de Markov, anteriormente mencionados.

 \cite{Kaeufer_Woitke_Min_Kamp_Pinte_2023} a través del análisis de las distribuciones de energía espectral (SEDs) de los discos protoplanetarios, estudiaron los parámetros estructurales de dichos discos, centrándose especialmente en la masa. Sin embargo, el estudio destacó dos factores importantes. En primer lugar, se señaló que el análisis de las distribuciones de energía espectral es altamente degenerado, lo que significa que diferentes configuraciones de parámetros pueden producir resultados muy similares en las SEDs. Para abordar esta cuestión, se propuso realizar un análisis bayesiano completo para modelar la incertidumbre de los parámetros. No obstante, cabe destacar que la obtención de las SEDs implica el uso de modelos de transferencia radiativa, los cuales pueden requerir varios minutos de CPU para simular cómo se propaga la radiación a través del disco. Por esta razón, se propuso una red neuronal para simular el proceso de generación de las SEDs, lo cual toma aproximadamente 1 ms en completarse. De este estudio, es importante destacar que se concluyó que existe una correlación entre algunos parámetros; en particular, los discos con altas masas tienen alturas de escala más bajas que aquellos con masas más bajas.

 Además, en su estudio, \cite{1994A&A...287..493T} presentaron un método para investigar el espacio de parámetros de discos protoplanetarios ajustando modelos teóricos a las observaciones de SEDs de FU Ori y DN Tau, con el objetivo de encontrar el mejor ajuste en un espacio de parámetros al minimizar la diferencia entre las observaciones y las predicciones del modelo (utilizando la métrica del chi-cuadrado). Para ello, emplearon un algoritmo de Metropolis, el cual es una técnica de MCMC, que permite explorar eficazmente un espacio de parámetros.

 En otro estudios, como el de \cite{Franceschi_Birnstiel_Henning_Pinilla_Semenov_Zormpas_2022} estudian la masa de discos protoplanetarios de observaciones provenientes de ALMA dada su importancia en el entendimiento de como los discos evolucionan a sistemas planetarios. Plantearon la hipótesis de que existe una relación entre la densidad superficial del disco y la ubicación de las lineas de polvo. Para llevar a cabo su estudio propusieron utilizar un modelo de evolución de polvo para probar dicha dependencia, el cual complementaron con las recomendaciones planteadas por \cite{2017ApJ...840...93P} sobre considerar los tiempos de deriva y el crecimiento para estimar la densidad superficial de gas en el borde exterior de la región emisora. Entonces, obtuvieron observaciones de la densidad superficial del disco a diferentes distancias radiales para ajustar el perfil auto-semejante de Lynden-Bell y Pringle a dichas observaciones para encontrar los parámetros del perfil, una vez encontrados se integra para obtener la masa. En cuanto a los resultados, concluyen que determinar la ubicación de la linea de polvo es un enfoque prometedor para la estimación de la masa, sin embargo, existe una gran dependencia de la estructura del disco y para trampas de polvo fuertes el modelo falla.

Como hemos podido evidenciar, se han desarrollado diversos métodos para estimar los parámetros estructurales de los discos protoplanetarios. Sin embargo, todos estos requieren el uso de modelos que de algún modo representen la transferencia radiativa, la cual es un proceso lento. Además, estos métodos consideran únicamente observaciones del radiotelescopio en una banda de frecuencia, y no existen modelos basados en inteligencia artificial para realizar la estimación de los parámetros en multiples bandas.

Sin embargo, existen estudios que han considerado el uso de Transformers para tareas de clasificación, ya que el uso de los \textit{self-attention modules} ha demostrado una gran capacidad para encontrar relaciones espaciales entre diferentes parches de una imagen previamente aplanada. Por ejemplo, \cite{2024arXiv240502218D} llevaron a cabo un estudio para identificar la hierba negra en cultivos de trigo y cebada mediante la comparación de tres modelos diferentes: una ResNet-50, EfficientNet B4 y un Transformer Swin-B. Este último es una mejora de los ViT, ya que introducen una arquitectura jerárquica que permite procesar imágenes en múltiples escalas de manera eficiente. El trabajo tiene como segundo objetivo demostrar el papel que tienen las diferentes bandas espectrales en el rendimiento de la clasificación de malezas. Y como resultados concluyeron que Swin-B fue el modelo que una mayor precisión del 87.7\%.

En otros estudios, \cite{Rouet-Leduc_Hulbert_2024} llevan a cabo un estudio de detección de emisiones de metano a partir de datos satelitales multiespectrales provenientes de Sentinel-2. La importancia de este estudio recae en la gran implicancia que tiene el gas metano en el calentamiento global, y poder detectarlo ayudaría enormemente a reducir sus emisiones dado que tiene una vida atmosférica más corta que la del CO\textsubscript{2}. En el estudio, \cite{Rouet-Leduc_Hulbert_2024} utilizaron un ViT para mejorar la detección de metano en datos multiespectrales. El ViT es utilizado para reemplazar el \textit{encoder} de una arquitectura U-Net; sin embargo, el \textit{decoder} mantiene las capas convolucionales de una U-Net. En cuanto a los resultados, el modelo presenta una tasa de falsos positivos del 0.03\%, lo cual es muy bajo.

Hemos podido evidenciar que existen diversos estudios que hacen uso de los Transformers para tareas relacionadas con imágenes; sin embargo, en el campo de la astronomía, son pocos los estudios que utilizan los ViT. \cite{lin2022galaxy} proponen por primera vez el uso de los ViT para la clasificación morfológica de galaxias y obtuvieron resultados prometedores, los cuales, para dicha tarea, son comparables con los que se pueden obtener con una CNN de última generación.



\subsection{Descripción de los datos o \textit{corpus}}

Para el presente estudio se van a utilizar discos protoplanetarios en el plano de visibilidades, es decir, plano \textit{(u,v)} que corresponden a las visibilidades grideadas que se representan como una matriz compleja. 

Además, para cada disco se generarán dos imágenes adicionales: una en la banda 4 y otra en la banda 8 del ALMA (Atacama Large Millimeter/submillimeter Array). Esto se hace con el propósito de permitir que el modelo sea capaz de reconocer los parámetros estructurales de un disco protoplanetario independientemente de la banda de frecuencia en la que haya sido capturado. Además, cada banda aporta información que permite observar de mejor o peor manera diferentes parámetros.

Dado que el método escogido hace uso de aprendizaje supervisado, a cada uno de los modelos de discos le corresponde un conjunto de parámetros, el cual contiene la descripción morfológica del disco, los principales parámetros que se buscan estimar son: masa total del disco $M_{d}$, gradiente de densidad superficial $\alpha$, radio característico $R_{c}$, la escala de altura normalizada $H_{100}$, ángulo de ensanchamiento del disco $\psi$, la inclinación $i$ y la posición del ángulo del eje mayor $PA$.